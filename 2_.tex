\documentclass[11pt]{article}
\usepackage{amsmath}
\usepackage{amssymb}
\begin{document}

LaTex document for LAFF class notes
\\
\\
Practice expressing statements as predicates
\\
\\
Let us consider a one dimensional array $b(1:n)$,
where $1$ $\leq$ $n$. Let j and k be two 
integer variables satisfying $1$ $\leq$ $j$ $ \leq$ $k$ $\leq$ $n$. by $b$ $(j:k)$ we mean the subarray of $b$
consisting of $b(j), b(j+1),...b(k)$. The segment $b(j:k)$ is empty(contains no elements) if $j > k$
\\
\\
Translate the sentences into predicates: 
\\
\\
Homework 1.4.5.1-1
\\
\\
All elements in the subarray $b(j:k)$ are positive.
\\
\\
$( \forall | j \leq i \leq k : b(i) > 0)$
\\
\\
Homework 1.4.5.1-2
\\
\\
$\neg(\exists i| j \leq i \leq k : b(i) > 0)$
\\
\\
$(\forall i | j \leq i \leq k : b(i) \leq 0)$
\\
\\
Homework 1.4.6.1-3
\\
\\
$\neg(\forall i| j \leq i \leq k : b(i) > 0)$
\\
\\
$(\exists i| j \leq i \leq k: b(i) \leq 0)$
\\
\\
Homework 1.4.6.1-4
\\
\\
All elements in the subarray $b(j : k)$ are not positive.
\\
\\
$\neg(\exists i| j \leq i \leq k : b(i) > 0)$
\\
\\
$(\forall i| j \leq i \leq k : b(i) \leq 0)$
\\
\\
Homework 1.4.6.1-5
\\
\\
$(\exists i|j \leq i \leq k : b(i) > 0)$
\\
\\
Homework 1.4.6.1-6
\\
\\
$(\exists i|j \leq i \leq k : b(i) > 0)$
\\
\\
Homework 1.4.6.1-7
\\
\\
$(\exists i | j \leq i \leq k : b(i) > 0)$
\\
\\
Homework 1.4.6.1-8
\\
\\
$\neg(\forall i | j \leq i \leq k : b(i) > 0)$
\\
\\
$(\exists i| j \leq i \leq k : b(i) \leq 0)$
\\
\\
Homework 1.4.6.1-9
\\
\\
$\neg(\forall i| j \leq i \leq k : b(i) > 0)$
\\
\\
$(\exists i | j \leq i \leq k : b(i) \leq 0)$
\\
\\
Homework 1.4.6.1-10
\\
\\
It is not the case that there is an element in the subarray $b(j:k)$ that is positive.
\\
\\
$\neg(\exists i| j \leq i \leq k : b(i) > 0)$
\\
\\
$(\forall i | j \leq i \leq k : b(i) \leq 0)$
\\
\\
Homework 1.4.6.2
\\
\\
Exactly one element in the subarray $b(j:k)$ is positive.
\\
\\
$(\exists i| j \leq i \leq k : b(i) > 0) \land (\forall p | j \leq p \leq k \land p \neq i : \neg (b(p) > 0)))$
\\
\\
\\
\\
\\
\\
Homework 1.4.6.3-1
\\
\\
\\
\\
A program specification is a "predicate" that describes the state of variables after the action of a program. It specifies what the program does and will be a very important first step in our work to derive programs. Write specifications for a program that...
\\
\\
Sets $s$ equal to the sum of the elements of $b(j:k)$.
\\
\\
$s = (\sum i|j \leq i \leq k : b(i))$
\\
\\
Homework 1.4.6.3-2
\\
\\
Sets $M$ equal to the maximum value in $b(j:k)$.
\\
\\
$(\exists i | j \leq i \leq k : M = b(i)) \land (\forall i | j \leq i \leq k : b(i) \leq M)$
\\
\\
Homework 1.4.6.3-3
\\
\\
Sets I equal to the index of a maximum value $b(j : k)$.
\\
\\
$(j \leq I \leq k) \land (\forall i | j \leq i \leq k : b(i) \leq b(I))$
\\
\\
Homework 1.4.6.3-4
\\
\\
$1 \leq x \leq n \land (\exists i | 0 \leq i : x = 2^i \land \frac{n}{2} < x \leq n)$
\\
\\
$1 \leq x \leq n \land (\exists i | 0 \leq i : x = 2^i) \land \neg(\exists i | x < 2^i < n)$
\\
\\
Homework 1.4.6.3-5
\\
\\
Computes $c$, the number of zeroes in array $b(1:n)$.
\\
\\
$(\sum i | 1 \leq i \leq n \land b(i) = 0 : 1)$
\\
\\
\\
\\
\\
\\
Homework 1.4.6.3-6
\\
\\
Consider array of integers $b(1:n)$. Each of its subsegments $b(i : j)$ has a sum $S_{i,j} = (\sum k | i \leq k \leq j : b(k))$. Compute $M$ equal to the maximum such sum. 
\\
\\
$(\exists i, j | 1 \leq i \leq j \leq n : M = S_{i, j}) \land \neg(\exists i, j | 1 \leq i \leq j \leq n : M < S_{i, j})$
\\
\\
$(\exists i, j|1 \leq i \leq j \leq n: M = S_{i,j}) \land (\forall i, j | 1 \leq i \leq j \leq n: M \geqslant S_{i, j})$
\\
\\



\end{document}