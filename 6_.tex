\documentclass{article}
\usepackage[utf8]{inputenc}

\title{The Hoare Triple}
\author{Krystal Maughan }
\date{April 27th 2017}

\begin{document}

\maketitle

\section{The Hoare Triple}
Homework 2.3.11
\\
\\
Consider the skip command, which simply doesn't do anything:
\\
\\
 $\left\{Q : ?\right\}$
\\
\\
skip
\\
\\
$\left\{R : x > 4\right\}$
\\
\\
From what state Q will the command skip finish?
\\
\\
(in a finite amount of time) in a state
\\
\\ where $ x > 4$ is TRUE?
\\
\\
In other words
\\
\\
wp("skip", $x > 4$) = 
\\
\\
$x > 4$
\\
\\
Explanation:
This is like the precondition for a loop.
$x$ must be more than 4 for the loop to begin (or end)
\\
\\
\\
\\
\\
\\
\\
\\
\\
\\
\\
Homework 2.3.1.2
\\
\\
Building on the last homework,
\\
consider the general case
\\
\\
$\left\{Q : ?\right\}$
\\
skip
\\
$\left\{R\right\}$
\\
\\
From what STATE will the command skip finish
\\
\\
(in a finite amount of time) 
\\
\\
in a state where $R$ is TRUE?
\\
\\
In other words
\\
\\
wp("skip", R) = 
\\
\\
R
\\
\\
Explanation: 
For a loop to commence or end,
\\
\\
it has to be in state R if R is the condition 
\\
\\
that must be satisfied to put the loop into motion
\\
\\
(or stop the loop).




\end{document}
